%!TEX root = ../dokumentation.tex

\RequirePackage[l2tabu, orthodox]{nag}	% weist in Commandozeile bzw. log auf veraltete LaTeX Syntax hin

\documentclass[%
    pdftex,
    oneside,			% Einseitiger Druck.
    12pt,				% Schriftgroesse
    parskip=half,		% Halbe Zeile Abstand zwischen Absätzen.
    %topmargin = 10pt,	% Abstand Seitenrand (Std:1in) zu Kopfzeile [laut log: unused]
    headheight = 33pt,	% Höhe der Kopfzeile
    %headsep = 30pt,	% Abstand zwischen Kopfzeile und Text Body  [laut log: unused]
    headsepline,		% Linie nach Kopfzeile.
    footsepline,		% Linie vor Fusszeile.
    %footheight = 16pt,	% Höhe der Fusszeile
    abstracton,		% Abstract Überschriften
    DIV=calc,		% Satzspiegel berechnen
    BCOR=8mm,		% Bindekorrektur links: 8mm
    headinclude=false,	% Kopfzeile nicht in den Satzspiegel einbeziehen
    footinclude=false,	% Fußzeile nicht in den Satzspiegel einbeziehen
    listof=totoc,		% Abbildungs-/ Tabellenverzeichnis im Inhaltsverzeichnis darstellen
    toc=bibliography,	% Literaturverzeichnis im Inhaltsverzeichnis darstellen
]{scrreprt}	% Koma-Script report-Klasse, fuer laengere Bachelorarbeiten alternativ auch: scrbook

% Einstellungen laden
\usepackage{xstring}
\usepackage{ifpdf}
\usepackage{ifluatex}

% Titleüberschriften
\usepackage{titlesec}
\titleformat{\chapter}
{\normalfont\huge\bfseries}{\thechapter}{20pt}{}
\titleformat{\section}
{\normalfont\Large\bfseries}{\thesection}{1em}{}
\titleformat{\subsection}
{\normalfont\large\bfseries}{\thesubsection}{1em}{}
\titleformat{\subsubsection}
{\normalfont\normalsize\bfseries}{\thesubsubsection}{1em}{}
\titleformat{\paragraph}[runin]
{\normalfont\normalsize\bfseries}{\theparagraph}{1em}{}
\titlespacing*{\chapter}     {0pt}{-40pt}{10pt}
\titlespacing*{\section}     {0pt}{10pt}{8pt}
\titlespacing*{\subsection}   {0pt}{8pt}{6pt}
\titlespacing*{\subsubsection}{0pt}{8pt}{6pt}
\titlespacing*{\paragraph}   {0pt}{8pt}{6pt}

\usepackage{lastpage}
\usepackage{fancyhdr}
\newcommand{\einstellung}[1]{%
    \expandafter\newcommand\csname #1\endcsname{}
    \expandafter\newcommand\csname setze#1\endcsname[1]{\expandafter\renewcommand\csname#1\endcsname{##1}}
}
\newcommand{\langstr}[1]{\einstellung{lang#1}}


\newcommand{\cellbreak}[3][c]{
	\begin{tabular}[#1]{@{}#2@{}}#3\end{tabular}
}


\input{ads/einstellungen_liste} % verfügbare Einstellungen
%%%%%%%%%%%%%%%%%%%%%%%%%%%%%%%%%%%%%%%%%%%%%%%%%%%%%%%%%%%%%%%%%%%%%%%%%%%%%%%
%                                   Einstellungen
%
% Hier k�nnen alle relevanten Einstellungen f�r diese Arbeit gesetzt werden.
% Dazu geh�ren Angaben u.a. �ber den Autor sowie Formatierungen.
%
%
%%%%%%%%%%%%%%%%%%%%%%%%%%%%%%%%%%%%%%%%%%%%%%%%%%%%%%%%%%%%%%%%%%%%%%%%%%%%%%%


%%%%%%%%%%%%%%%%%%%%%%%%%%%%%%%%%%%% Sprache %%%%%%%%%%%%%%%%%%%%%%%%%%%%%%%%%%%
%% Aktuell sind Deutsch und Englisch unterst�tzt.
%% Es werden nicht nur alle vom Dokument erzeugten Texte in
%% der entsprechenden Sprache angezeigt, sondern auch weitere
%% Aspekte angepasst, wie z.B. die Anf�hrungszeichen und
%% Datumsformate.
\setzesprache{de} % de oder en
%%%%%%%%%%%%%%%%%%%%%%%%%%%%%%%%%%%%%%%%%%%%%%%%%%%%%%%%%%%%%%%%%%%%%%%%%%%%%%%%

%%%%%%%%%%%%%%%%%%%%%%%%%%%%%%%%%%% Angaben  %%%%%%%%%%%%%%%%%%%%%%%%%%%%%%%%%%%
%% Die meisten der folgenden Daten werden auf dem
%% Deckblatt angezeigt, einige auch im weiteren Verlauf
%% des Dokuments.
\setzemartrikelnr{9275184}
\setzekurs{TINF19ITA}
\setzetitel{Implementierung von Set-Of-Support-Strategien f\"ur das Theorem-Beweisen in Python}
\setzedatumAnfang{29.09.2021}
\setzedatumAbgabe{10.06.2022}
\setzefirma{Robert Bosch GmbH}
\setzefirmenort{Stuttgart}
\setzeabgabeort{Stuttgart}
\setzeabschluss{Bachelor of Science}
\setzestudiengang{Informatik}
\setzedhbw{Stuttgart}
\setzebetreuer{Professor Dr. Stephan Schulz}
\setzegutachter{}
\setzezeitraum{29.09.2021 - 10.06.2022}
\setzearbeit{Studienarbeit}
\setzeautor{Nico Makowe}
\setzesemester{4}
\setzestudienrichtung{IT-Automotive}
\setzejahrgang{2019}
\setzeabteilung{}
\setzestandort{Feuerbach}


%\reviewertrue				% auskommentiren oder �ndern zu \reviewerfalse wenn kein Gutachter gesetzt werden muss

% Angabe des roten/gelben Punktes auf dem Titelblatt zur Kennzeichnung der Vertraulichkeitsstufe.
% M�gliche Angaben sind \yellowdottrue und \reddottrue. Werden beide angegeben, wird der rote Punkt gezeichnet.
% Wird keines der Kommandos angegeben, wird kein Punkt gezeichnet
%\yellowdottrue

%%%%%%%%%%%%%%%%%%%%%%%%%%%%%%%%%%%%%%%%%%%%%%%%%%%%%%%%%%%%%%%%%%%%%%%%%%%%%%%%

%%%%%%%%%%%%%%%%%%%%%%%%%%%%%%%%% Layout %%%%%%%%%%%%%%%%%%%%%%%%%%%%%%%%%%%%%%%
%% Verschiedene Schriftarten
% laut nag Warnung: palatino obsolete, use mathpazo, helvet (option scaled=.95), courier instead
\setzeschriftart{lmodern} % palatino oder goudysans, lmodern, libertine

%% Abstand vor Kapitel�berschriften zum oberen Seitenrand
\setzekapitelabstand{10pt}

%% Spaltenabstand
\setzespaltenabstand{10pt}
%%Zeilenabstand innerhalb einer Tabelle
\setzezeilenabstand{1.5}
%%%%%%%%%%%%%%%%%%%%%%%%%%%%%%%%%%%%%%%%%%%%%%%%%%%%%%%%%%%%%%%%%%%%%%%%%%%%%%%% % lese Einstellungen

\input{lang/strings} % verfügbare Strings
\input{lang/\sprache} % Übersetzung einlesen


%\lstset{language=Matlab}
\newcommand{\citem}[1]{\item[\texttt{#1}]} % Code-Item für description-Liste
%\newcommand\todo[1]{\textit{\textcolor{red}{TODO: #1}}\message{LaTeX Warning: \noexpand TODO item left in line \the\inputlineno}} % Todo-Item
\newcommand\todo[1]{\textit{\textcolor{red}{TODO: #1}}} % Todo-Item
\usepackage{pdfpages}         % pdf-Seiten einbinden

%% Farben (Angabe in HTML-Notation mit großen Buchstaben)
\newcommand{\ladefarben}{%
	\definecolor{LinkColor}{HTML}{00007A}
	\definecolor{ListingBackground}{HTML}{FCF7DE}
}
%% Mathematikpakete benutzen (Pakete aktivieren)
%\usepackage{amsmath}
%\usepackage{amssymb}

%% Programmiersprachen Highlighting (Listings)
\newcommand{\listingsettings}{%
	\lstset{%
		language=Python,		% Standardsprache des Quellcodes
		numbers=left,			% Zeilennummern links
		stepnumber=1,			% Jede Zeile nummerieren.
		numbersep=5pt,			% 5pt Abstand zum Quellcode
		numberstyle=\tiny,		% Zeichengrösse 'tiny' für die Nummern.
		breaklines=true,		% Zeilen umbrechen wenn notwendig.
		breakautoindent=true,	% Nach dem Zeilenumbruch Zeile einrücken.
		postbreak=\space,		% Bei Leerzeichen umbrechen.
		tabsize=4,				% Tabulatorgrösse 2
		basicstyle=\ttfamily\footnotesize, % Nichtproportionale Schrift, klein für den Quellcode
		showspaces=false,		% Leerzeichen nicht anzeigen.
		showstringspaces=false,	% Leerzeichen auch in Strings ('') nicht anzeigen.
		extendedchars=true,		% Alle Zeichen vom Latin1 Zeichensatz anzeigen.
		captionpos=b,			% sets the caption-position to bottom
		%backgroundcolor=\color{ListingBackground}, % Hintergrundfarbe des Quellcodes setzen.
		xleftmargin=2em,		% Rand links
		xrightmargin=0pt,		% Rand rechts
		frame=single,			% Rahmen an
		frameround=ffff,
		rulecolor=\color{darkgray},	% Rahmenfarbe
		framexleftmargin=1.5em,	% Rahmenrand sodass Zahlennummern innerhalb
		%fillcolor=\color{ListingBackground},
		keywordstyle=\color[rgb]{0.133,0.133,1.0},
		commentstyle=\color[rgb]{0.133,0.545,0.133},
		stringstyle=\color[rgb]{0.627,0.126,0.941},
    aboveskip=1.5em,
	}
}





%%%%%%%%%%%%%%%%%%%%%%%%%%%%% Kopf-/Fußzeilenwechsel %%%%%%%%%%%%%%%%%%%%%%%%%%%
\setlength{\headheight}{40pt}

\newcommand{\setpagestylehead}{%
    \fancypagestyle{plain}{%
        \fancyhf{}
        \fancyhead[L]{\vspace{0.5cm}\small \langkopfz}
        \fancyhead[R]{
            \hspace{2.0cm}
            %trim=left bottom right top
            \iflang{de}{
            	\begin{textblock*}{188mm}(-5mm,9mm)            	
            	\includegraphics[height=1.4cm]{images/dhbw_de}
            	\end{textblock*}
            	}
        }
        \fancyfoot[L]{
            \noindent{\tiny \langfussz\\
                \begin{tabular*}{16cm}{@{\extracolsep{\fill}}l>{\raggedleft}p{8cm}}
                    {\tiny \langstand: \today} & 
                    {\tiny \langseite\ \thepage\ \langseitevon\ \pageref*{endOfRomanNumbering}\vspace{1cm}}\tabularnewline
                \end{tabular*}
            }
        }
    }
    \pagestyle{plain}
    \pagenumbering{roman}
}    

\newcommand{\setpagestylecontent}{
\fancypagestyle{plain}{%
        \fancyhf{}
        \fancyhead[L]{\vspace{0.5cm}\small \langkopfz}
        \fancyhead[R]{
            \hspace{2.0cm}
            %trim=left bottom right top
            \iflang{de}{
            	\begin{textblock*}{188mm}(-5mm,9mm)            	
            		\includegraphics[height=1.4cm]{images/dhbw_de}
            	\end{textblock*}
            }
            \iflang{en}{
            	\begin{textblock*}{188mm}(-5mm,7mm)            	
            		\includegraphics[height=1.2cm]{images/dhbw_en}
            	\end{textblock*}
            }
        }
        \fancyfoot[L]{
            \noindent{\tiny \langfussz\\
                \begin{tabular*}{16cm}{@{\extracolsep{\fill}}l>{\raggedleft}p{8cm}}
                    {\tiny \langstand: \today} & 
                    {\tiny \langseite\ \thepage\ \langseitevon\ \pageref*{endOfArabicNumbering}\vspace{1cm}}\tabularnewline
                \end{tabular*}
            }
        }
    }
    \pagestyle{plain}
    \pagenumbering{arabic}
}

\newcommand{\setpagestylefoot}{
\fancypagestyle{plain}{%
        \fancyhf{}
        \fancyhead[L]{\vspace{0.5cm}\small \langkopfz}
        \fancyhead[R]{
            \hspace{2.0cm}
            %trim=left bottom right top
            \iflang{de}{
            	\begin{textblock*}{188mm}(-5mm,9mm)            	
            		\includegraphics[height=1.4cm]{images/dhbw_de}
            	\end{textblock*}
            }
            \iflang{en}{
            	\begin{textblock*}{188mm}(-5mm,7mm)            	
            		\includegraphics[height=1.2cm]{images/dhbw_en}
            	\end{textblock*}
            }
        }
        \fancyfoot[L]{
            \noindent{\tiny \langfussz\\
                \begin{tabular*}{16cm}{@{\extracolsep{\fill}}l>{\raggedleft}p{8cm}}
                    {\tiny \langstand: \today} & 
                    {\tiny \langseite\ \thepage\ \langseitevon\ \pageref*{LastPage}\vspace{1cm}}\tabularnewline
                \end{tabular*}
            }
        }
    }
    \pagestyle{plain}
    \pagenumbering{Alph}
}


%%%%%%%%%%%%%%%%%%%%%%%%%%%%%%%%%%%%%%%%%%%%%%%%%%%%%%%%%%%%%%%%%%%%%%%%%%%%%%%%

% Einstellung der Sprache des Paketes Babel und der Verzeichnisüberschriften

\iflang{de}{
    \usepackage[english, ngerman]{babel}
    \selectlanguage{ngerman}
}
\iflang{en}{
    \usepackage[ngerman, english]{babel}
    \selectlanguage{english}
}

\usepackage[utf8]{inputenc}
\usepackage[T1]{fontenc}
\usepackage{tikz}
\usepackage{xcolor}
\usepackage{additionalPackages/tikz-uml} % UML Diagramme
\usepackage[european]{additionalPackages/circuitikz}
%%%%%%% Package Includes %%%%%%%

\usepackage[margin=2.5cm,foot=1cm,top=3cm,bottom=3cm]{geometry}	% Seitenränder und Abstände
\usepackage[activate]{microtype} %Zeilenumbruch und mehr
\usepackage[onehalfspacing]{setspace}
\usepackage{makeidx}
\usepackage[autostyle=true,german=quotes]{csquotes}
\usepackage{longtable}
\usepackage{enumitem}	% mehr Optionen bei Aufzählungen
\usepackage{graphicx}
\usepackage{xcolor} 	% für HTML-Notation
\usepackage{float}
\usepackage{array}
\usepackage{calc}		% zum Rechnen (Bildtabelle in Deckblatt)
\usepackage[right]{eurosym}
\usepackage{wrapfig}
\usepackage{pgffor} % für automatische Kapiteldateieinbindung
\usepackage[perpage, hang, multiple, stable]{footmisc} % Fussnoten
\usepackage{acronym}
\usepackage[absolute]{textpos}
%\usepackage[printonlyused, footnote]{acronym} % falls gewünscht kann die Option footnote eingefügt werden, dann wird die Erklärung nicht inline sondern in einer Fußnote dargestellt
\usepackage{scrhack} % in Kombination mit listings-Package kommt es zu Warnings, dieses Paket verhindert die Warnings! Ggf. auskommentieren und die Warnings akzeptieren falls Verzeichnisse nicht so dargestellt werden wie gewünscht
\usepackage{listings} % Code-Listings
%\usepackage[numbered, framed]{matlab-prettifier}
%\usepackage[framed]{matlab-prettifier} % .sty-Datei muss vorhanden sein! Kann auskommentiert werden, falls keine Matlab-Listings in der Arbeit enthalten sind.
\usepackage{color, colortbl}  %Für Highlighten der Tabellenzeilen
\usepackage{amsmath}% http://ctan.org/pkg/amsmath
\usepackage{color}
\usepackage[nottoc]{tocbibind}
\usepackage[backend=bibtex]{biblatex}
\usepackage{multirow}

\addbibresource{bibliographie.bib}

% eine Kommentarumgebung "k" (Handhabe mit \begin{k}<Kommentartext>\end{k},
% Kommentare werden rot gedruckt). Wird \% vor excludecomment{k} entfernt,
% werden keine Kommentare mehr gedruckt.
\usepackage{comment}
\specialcomment{k}{\begingroup\color{red}}{\endgroup}
%\excludecomment{k}


%%%%%% Configuration %%%%%

%% Anwenden der Einstellungen

\usepackage{\schriftart}
\ladefarben{}

% Titel, Autor und Datum
\title{\titel}
\author{\autor}
\date{\datum}

%\usepackage[list=true]{subcaption}

% PDF Einstellungen
\usepackage[%
    pdftitle={\titel},
    pdfauthor={\autor},
    pdfsubject={\arbeit},
    pdfcreator={pdflatex, LaTeX with KOMA-Script},
    pdfpagemode=UseOutlines, 		% Beim Oeffnen Inhaltsverzeichnis anzeigen
    pdfdisplaydoctitle=true, 		% Dokumenttitel statt Dateiname anzeigen.
    pdflang={\sprache}, 			% Sprache des Dokuments.
]{hyperref}

% (Farb-)einstellungen für die Links im PDF
\hypersetup{%
    colorlinks=true, 		% Aktivieren von farbigen Links im Dokument
    linkcolor=black, 	    % Farbe festlegen
    citecolor=LinkColor,
    filecolor=LinkColor,
    menucolor=LinkColor,
    urlcolor=LinkColor,
    %linktocpage=true, 		% Nicht der Text sondern die Seitenzahlen in Verzeichnissen klickbar
    linktoc=all,            % Seitenzahlen und Text klickbar
    bookmarksnumbered=true 	% Überschriftsnummerierung im PDF Inhalt anzeigen.
}
% Workaround um Fehler in Hyperref, muss hier stehen bleiben
\usepackage{bookmark} %nur ein latex-Durchlauf für die Aktualisierung von Verzeichnissen nötig

% Schriftart in Captions etwas kleiner
\addtokomafont{caption}{\small}

\usepackage{subfig}


% Glossar
\usepackage[nonumberlist,toc]{glossaries}
\usepackage{blindtext} % Blindtext-Package. Common Usage: \blindtext für einzelnen Abschnitt, \Blindtext für mehrere Abschnitte

%%%%%% Additional settings %%%%%%

% Hurenkinder und Schusterjungen verhindern
% http://projekte.dante.de/DanteFAQ/Silbentrennung
\clubpenalty = 10000 % schließt Schusterjungen aus (Seitenumbruch nach der ersten Zeile eines neuen Absatzes)
\widowpenalty = 10000 % schließt Hurenkinder aus (die letzte Zeile eines Absatzes steht auf einer neuen Seite)
\displaywidowpenalty=10000

% Bildpfad
\graphicspath{{images/}}

% Einige häufig verwendete Sprachen
\lstloadlanguages{Python}
\listingsettings{}
% Umbennung des Listings
\renewcommand\lstlistingname{\langlistingname}
\renewcommand\lstlistlistingname{\langlistlistingname}
\def\lstlistingautorefname{\langlistingautorefname}

% Abstände in Tabellen
\setlength{\tabcolsep}{\spaltenabstand}
\renewcommand{\arraystretch}{\zeilenabstand}

\usepackage{xspace}
\newcommand{\lastcontentpage}{}
\usepackage{amsfonts}

\usetikzlibrary{shapes,arrows,calc}
\usepackage{relsize}

\usepackage{censor}

\usepackage{eso-pic}


%RJG8FE: add a pageref to autoref whenever the referenced page is not the same as the current one
%        useful for printed documents without clickable hyperlinks
\AtBeginDocument{\let\oldautoref\autoref}
\AtBeginDocument{
    \renewcommand{\autoref}[1]{%
        \oldautoref{#1}%
        \ifthenelse{\thepage=\pageref{#1}}% if current page number equals the referenced page number
        {}% then add nothing
        { (S. \pageref{#1})}% else add the text
    }
}

\usepackage{amssymb} % Erweiterung der Symbole in Mathematikumgebung

\iflang{de}{\usepackage{icomma}} % Europäsiches Komma in Formeln

\lstdefinelanguage{TypeScript}{
	keywords={break, case, catch, continue, debugger, default, delete, do, else, false, finally, for, function, if, in, instanceof, new, null, return, switch, this, throw, true, try, typeof, var, void, while, with, type string, number, constructor, Person, string, let, any, T},
	morecomment=[l]{//},
	morecomment=[s]{/*}{*/},
	morestring=[b]',
	morestring=[b]",
	ndkeywords={class, export, boolean, throw, implements, import, this},
	keywordstyle=\color{blue}\bfseries,
	ndkeywordstyle=\color{purple}\bfseries,
	identifierstyle=\color{black},
	commentstyle=\color{purple}\ttfamily,
	stringstyle=\color{red}\ttfamily,
	sensitive=true,
	numbers=left,
	stepnumber=1
}

\definecolor{cyan}{RGB}{0,255,255}


