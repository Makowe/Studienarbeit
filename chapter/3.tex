%!TEX root = ../dokumentation.tex

\chapter{Konzeption}

\section{Separierung von SOS-Klauseln und Nicht-SOS-Klauseln}

PyRes erzeugt bei jeder Beweisprozedur ein Objekt der Klasse ProofState (Beweiszustand). Innerhalb dieses Objekts sind die verarbeiteten und unverarbeiteten Klauseln als zwei Klauselmengen gespeichert. Um SOS-Strategien implementieren zu können, müssen Klauseln des SOS und Klauseln, die nicht zum SOS gehören, unterscheidbar gemacht werden, um sie unterschiedlich behandeln zu können. Dafür werden zwei Konzepte in Betracht gezogen.
\begin{enumerate}
	\item Anstatt den bisher zwei Klauselmengen für verarbeitete und unverarbeitete Klauseln, werden drei Klauselmengen eingesetzt. Jeweils eine Klauselmenge enthält die verarbeiteten Klauseln, die unverarbeiteten SOS-Klauseln und die unverarbeiteten Nicht-SOS-Klauseln.
	\item Jede Klausel erhält eine Zusatzinformation in Form eines Wahrheitswertes, der aussagt, ob sich eine Klausel im SOS befindet oder nicht. Der Wahrheitswert ist als Markierung der Klausel zu verstehen. Das ProofState-Objekt speichert SOS-Klauseln und Nicht-SOS-Klauseln gemischt innerhalb der verarbeiteten und unverarbeiteten Klauselmengen ab.
\end{enumerate}
Beide Lösungsansätze haben Vor- und Nachteile. Ein Vorteil der ersten Variante ist, dass die Klauseln auf Programmebene getrennt voneinander sind. Durch diese Abkapselung könnte es später einfacher und schneller sein, eine neue SOS-Klausel auszuwählen. Diese Variante ist auch einfacher zu verstehen und intuitiver, da die zwei separaten Klauselmengen der theoretischen Grundlage zweier getrennter Mengen entspricht. Ein Nachteil ist, dass der Beweismechanismus stark angepasst werden muss. Zum Beispiel muss für die Resolution eine SOS-Klausel nicht mehr nur mit den verarbeiteten Klauseln kombiniert werden, sondern auch mit den Nicht-SOS-Klauseln. In Hinblick auf die weitere Umsetzung wird es vermutlich Sonderfälle geben, bei denen Nicht-SOS-Klauseln verarbeitet werden müssen. Für die Implementierung solcher Sonderfälle sind zwei getrennte Klauselmengen eher eine Hürde, da die Kombination verschiedener Klauseln komplexer wird. Ein weiterer Nachteil der ersten Variante ist, dass die Information, ob eine Klausel im SOS ist, nicht direkt an die Klausel gebunden ist. Eine Methode, die eine Klausel übergeben bekommt, kann somit nicht prüfen, ob die übergebene Klausel im SOS ist. Der Methode müsste deshalb ein zusätzlicher Wahrheitswert übergeben werden, was zu größeren Anpassungen führt.
Für die Implementierung wurde sich nach Abwägung der Vor- und Nachteile für Variante 2 entschieden. Das Hauptargument ist, dass der Code weniger stark angepasst werden muss, aber trotzdem eine einfache Unterscheidungsmöglichkeit eingeführt wird.


Aufteilungsmöglichkeiten in Grundklauselmenge und SOS, SOS durch Flags in Klauseln darstellen, Strategie, um neue Klausel auszuwählen (Non-SOS in processed, oder alternierend mit Verhältnis),
Sonderfälle: Wenn SOS leer, 