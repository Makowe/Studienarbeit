%!TEX root = ../dokumentation.tex

\chapter{Zusammenfassung und Ausblick}

In dieser Arbeit wurden SOS-Strategien für den Theorembeweiser PyRes in der Sprache Python implementiert. Der Benutzer kann eine von vier Strategien auswählen: (1) Negierte Vermutung als SOS, (2) rein positive Klauseln oder (3) rein negative Klauseln als SOS und eine optimale Strategie, die anhand der Vorhandenheit von Vermutungen entweder Strategie 1 oder 3 anwendet. Weiterhin ist es möglich eine von zwei Verarbeitungskonzepten auszuwählen. Beim ersten Konzept werden alle Nicht-SOS-Klauseln als verarbeitet angesehen und nur SOS-Klauseln werden verarbeitet. Beim zweiten Konzept werden SOS- und Nicht-SOS-Klauseln mit unterschiedlichen Prioritäten verarbeitet.

Die SOS-Strategien steigern die Effizienz der Beweissuche deutlich. Am besten schneidet Strategie 1 in Verbindung mit SOS-Konzept 2 ab. Das ideal Verhältnis, in dem SOS-Klauseln im Vergleich zu Nicht-SOS-Klauseln verarbeitet werden, liegt in der Größenordnung von XXXXXXX. SOS-Konzept 2 ist in Verbindung mit anderen Optimierungsstrategien wie Literalselektion oder geordneter Resolution vollständig und erhöht die Performance der Beweissuche erneut. 



Ziel und Sinn der Aufgabe erreicht,
Errungenschaften, Kritik,
Ausblick