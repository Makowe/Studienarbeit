%!TEX root = ../dokumentation.tex

\chapter{Zusammenfassung und Ausblick}

In dieser Arbeit wurden Set-Of-Support-Strategien (SOS-Strategien) für den Theorembeweiser PyRes in der Sprache Python implementiert und verglichen. Der Benutzer kann eine von vier Strategien auswählen: (1) Negierte Vermutung als SOS, (2) rein positive Klauseln oder (3) rein negative Klauseln als SOS und eine optimale Strategie, die anhand der Vorhandenheit von Vermutungen entweder Strategie 1 oder 3 anwendet. Weiterhin ist es möglich eine von zwei Verarbeitungskonzepten auszuwählen. Beim ersten Konzept werden alle Nicht-SOS-Klauseln als verarbeitet angesehen und nur SOS-Klauseln werden verarbeitet. Beim zweiten Konzept werden SOS- und Nicht-SOS-Klauseln mit unterschiedlichen Prioritäten verarbeitet. Das zweite Konzept 2 ist in Verbindung mit anderen Optimierungsstrategien wie Literalselektion oder geordneter Resolution vollständig. 

Die SOS-Strategien steigern die Effizienz der Beweissuche deutlich. Am besten schneidet Strategie 1 in Verbindung mit SOS-Konzept 2 und negativer Literalselektion ab. Das ideal Verhältnis, in dem SOS-Klauseln im Vergleich zu Nicht-SOS-Klauseln verarbeitet werden, liegt in der Größenodrnung zwischen 2 und 4. 

PyRes ist anderen Beweiser wie E aufgrund der schlechteren Performance von Python gegenüber C deutlich unterlegen. Da alle Ergebnisse mit Konfigurationen ohne SOS als Referenzgrößen verglichen wurden, sind die gewonnenen Erkenntnisse dennoch auf andere Theorembeweiser übertragbar. E setzt die Set-Of-Support-Strategien bereits ein. Trotzdem könnte die Performance des Beweisers durch Anpassen des Algorithmus zur Klauselasuwahl geringfügig verbessert werden, denn wie in der Arbeit gezeigt wurde, hat eine kleine Veränderung in der Priorisierung der Klauseln, signifikante Auswirkungen auf den Erfolg der Beweissuche.

Das strikte Vorgehen nach SOS ist in der Praxis eher nicht nützlich, da es mit manchen anderen Strategie nicht kombinierbar ist. Die flexiblere Variante, welche die Vollständigkeit auch in Kombination mit anderen Strategien erhält, ist in der Praxis deutlich mächtiger und wird vermutlich auch in Zukunft weiterhin eingesetzt werden.