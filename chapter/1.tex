%!TEX root = ../dokumentation.tex

\chapter{Einleitung}
\section{Problemstellung}
Beim Testen von Software können Theorembeweiser eingesetzt werden, um Anforderungen zu verifizieren. Diese Beweiser können mithilfe gegebener Aussagen neue Zusammenhänge und auch Widersprüche herleiten.

Ein typisches Beispiel für den Einsatz von Theorembeweisern ist die Verifikation von Gleitkommarechenwerken \cite{Harrison2006Float}. Sie kommen in fast allen modernen Prozessoren vor. Die Schwierigkeit bei Gleitkommazahlen liegt darin, dass reelle Zahlen mit beliebiger Größe von einer Bitsequenz repräsentiert werden soll, die eine begrenzte Größe besitzt. Die Zahlen sind deshalb als Potenz kodiert, was sowohl für sehr große als auch für sehr kleine Zahlen eine präzise Abstufung ermöglicht. Für aufwändige Berechnungen wie Division, werden die Algorithmen entsprechend komplex. Nachdem bei Intel-Prozessoren im Jahr 1994 ein Hardwarefehler entdeckt worden war, der zu Fehlern mancher Divisionen führte, wurden bei den Nachfolgermodellen Theorembeweiser eingesetzt, um die Korrektheit der neuen Algorithmen mathematisch zu beweisen. \cite{WillkommIntel}

Es gibt Beweisverfahren, die vollständig fürs Widerlegen sind. Das heißt, dass es aus einer widersprüchlichen Menge von Aussagen, immer möglich ist, einen Widerspruch nachzuweisen. In der Praxis wird dieses Ziel nicht immer erreicht, da häufig unendlich viele Formeln abgeleitet werden können und es bei wachsender Formelmenge immer unwahrscheinlicher wird, einen Widerspruch zu finden. Es gibt verschiedene Ansätzen, um dem schnellen Anwachsen der Formelmenge entgegenzuwirken. Zum Beispiel kann die Anzahl der gegebenen Aussagen verringert werden, wenn bestimmtes Wissen nicht zum Beweis beiträgt, oder es können Verfahren zur Reduzierung von Formeln eingesetzt werden. Bei der Set-of-Support-Strategie (SOS) werden die gegebenen Aussagen in zwei Formelmengen eingeteilt und es werden Regeln definiert, die die Ableitung neuer Formeln einschränken. Auf diese Weise können weniger neue Formeln ableiten werden können, wodurch die Wahrscheinlichkeit, einen Widerspruch zu finden, erhöht wird. Der gesamte Beweisprozess soll mit der SOS-Strategie sowohl effektiver werden, als auch effizienter.

Das Ziel dieser Arbeit ist die Konzeption und Implementierung von Set-Of-Support-Strategien im bestehenden Theorembeweiser "PyRes". Außerdem soll das Laufzeitverhalten der Beweisführung mit und ohne SOS-Strategie verglichen werden. Ein weiteres Ziel ist, die Implementierung mit anderen Optimierungsstrategien, wie Literal-Selektion und geordneter Resolution kompatibel zu machen.

\section{Vorgehensweise der Arbeit}

Im Kapitel Grundlagen wird theoretisches Basiswissen vermittelt, das für das Verständnis dieser Arbeit notwendig ist. Der Fokus liegt auf der Prädikatenlogik erster Stufe und dem Resolutionsverfahren zur Beweisführung. Außerdem wird ein Überblick über die bestehende Version des Programms PyRes geschaffen.
Der Schwerpunkt bei der Konzeption liegt darauf, den formalen Ablauf der Beweisführung herauszuarbeiten und zu planen. Das fertige Konzept wird im Teil Implementierung umgesetzt und getestet.
Die Validierung bewertet die Verbesserung des Programms mit dem alten Stand ohne SOS-Strategie. Die Bewertung basiert auf die Ergebnisse von Performancetests, die das Laufzeit der Beweisführungen vergleichen.
Alle Ergebnisse werden in einer kritischen Reflexion zusammengefasst und reflektiert. Außerdem wird ein Ausblick für die Zukunft gegeben.