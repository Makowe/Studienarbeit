%!TEX root = ../dokumentation.tex

\chapter{Einleitung}
\section{Problemstellung}
Beim Testen von Software können Theorembeweiser eingesetzt werden, um Anforderungen zu verifizieren. Diese Beweiser können mithilfe gegebener Aussagen neue Zusammenhänge und auch Widersprüche herleiten.

Ein typisches Beispiel für den Einsatz von Theorembeweisern ist die Verifikation von Gleitkommarechenwerken \cite{Harrison2006Float}. Sie kommen in fast allen modernen Prozessoren vor. Die Schwierigkeit bei Gleitkommazahlen liegt darin, dass die unendlich große Menge reeller Zahlen auf Bitsequenzen mit begrenzter Größe abgebildet werden soll. Die Zahlen sind deshalb als Potenz kodiert, was sowohl für sehr große als auch für sehr kleine Zahlen eine präzise Abstufung ermöglicht. Die Algorithmen, die in Hardware implementiert sind, werden für Operationen wie Division entsprechend komplex. Seitdem bei Intel-Prozessoren im Jahr 1994 ein Hardwarefehler entdeckt worden war, der zu Fehlern mancher Berechnungen führte, werden Theorembeweiser eingesetzt, um die Korrektheit der Gleitkommawerke mathematisch zu beweisen. \cite{WillkommIntel} 

Es gibt Beweisverfahren, die vollständig fürs Widerlegen sind. Das heißt, aus einer widersprüchlichen Menge von Aussagen, ist es immer möglich, einen Widerspruch nachzuweisen. In der Praxis wird dieses Ziel nicht immer erreicht, da häufig unendlich viele Formeln abgeleitet werden können und es bei wachsender Formelmenge immer unwahrscheinlicher wird, einen Widerspruch zu finden. Es gibt verschiedene Ansätzen, um dem schnellen Anwachsen der Formelmenge entgegenzuwirken. Bei den Set-of-Support-Strategien (SOS-Strategien) werden die gegebenen Aussagen in zwei Formelmengen eingeteilt und es werden Regeln definiert, die die Ableitung neuer Formeln einschränken. Auf diese Weise können weniger neue Formeln ableiten werden können, wodurch die Wahrscheinlichkeit, einen Widerspruch zu finden, erhöht werden soll. Die benötigte Zeit für einen Beweis soll möglichst verringert werden.
Das Ziel dieser Arbeit ist die Konzeption und Implementierung von Set-Of-Support-Strategien im bestehenden Theorembeweiser "PyRes". Außerdem soll das Laufzeitverhalten der Beweisführung mit und ohne SOS-Strategie verglichen werden. Die Implementierung soll mit anderen Optimierungsstrategien, wie Literal-Selektion und geordneter Resolution kompatibel sein.

\section{Vorgehensweise der Arbeit}

Im Kapitel Grundlagen wird theoretisches Basiswissen vermittelt, das für das Verständnis dieser Arbeit notwendig ist. Der Fokus liegt auf der Prädikatenlogik erster Stufe, dem Resolutionsverfahren und dem theoretischen Prinzip der Set-Of-Support-Strategien. Außerdem wird ein Überblick über die bestehende Version des Programms PyRes und das eingesetzte Benchmark TPTP geschaffen.
Der Schwerpunkt bei der Konzeption liegt darauf, den formalen Ablauf der Beweisführung herauszuarbeiten und zu planen. Es sollen verschiedene Lösungsansätze verglichen und abgewogen werden. Das fertige Konzept wird im Teil Implementierung umgesetzt und getestet.
In der Validierung wird das Laufzeitverhalten von PyRes mit und ohne Set-Of-Support-Strategien verglichen. Auch die Verständlichkeit und Erweiterbarkeit der Implementierung wird bei der Bewertung mitbeachtet, da die einfache Codestruktur eine wichtige Charakteristik von PyRes ist. Offene Fragen, die in der Zukunft geklärt werden können und potenzielle Erweiterungsmöglichkeiten werden in einem abschließenden Ausblick erwähnt.